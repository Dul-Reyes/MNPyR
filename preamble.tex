\usepackage[utf8]{inputenc}% Usa codificación 8-bit que tiene 256 glyphs
\usepackage[T1]{fontenc}
%Más información de babel : http://www.texnia.com/spanishopt.html
%es-tabla: Traduce table como tabla en lugar de como cuadro.
%es-nodecimaldot: No añade punto tras los números de sección, subsección
\usepackage[spanish,es-nodecimaldot,es-tabla]{babel} 
\usepackage{graphicx} %Manipulación de imágenes : https://es.overleaf.com/learn/latex/Inserting_Images
\usepackage{tikz} % Requerido para dibujar formas personalizadas
\usepackage{float} %Gráficas
\usepackage{fancyhdr} %Para un mejor manejo de encabezados
\usepackage{calc} %Para hacer cálculos matemáticos en Latex
\usepackage{amsmath,amssymb,amsfonts} %Para tener mejores opciones en estructuras matemáticas.
\usepackage{amsthm} %Ambientes matemáticos : http://www.ctex.org/documents/packages/math/amsthdoc.pdf
%Vamos a colocar todo el formato que yihui colocó en su bookdown:https://github.com/rstudio/bookdown/blob/master/inst/examples/latex/preamble.tex
\usepackage{mathrsfs}
\usepackage{booktabs}
\usepackage{longtable}
\renewcommand{\textfraction}{0.05}
\renewcommand{\topfraction}{0.8}
\renewcommand{\bottomfraction}{0.8}
\renewcommand{\floatpagefraction}{0.75}
\usepackage{listings}
\lstset{
  breaklines=true
}